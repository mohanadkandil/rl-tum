%%%%%%%%%%%%%%%%%%%%%%%%%%%%%%%%%%%%%%%%%%%%%%%%%%%%%%%%%%%%%%%%%%%%%%%%%%%%%%%%
% TUM-Vorlage: Präsentation
%%%%%%%%%%%%%%%%%%%%%%%%%%%%%%%%%%%%%%%%%%%%%%%%%%%%%%%%%%%%%%%%%%%%%%%%%%%%%%%%
%
% Rechteinhaber:
%     Technische Universität München
%     https://www.tum.de
% 
% Gestaltung:
%     ediundsepp Gestaltungsgesellschaft, München
%     http://www.ediundsepp.de
% 
% Technische Umsetzung:
%     eWorks GmbH, Frankfurt am Main
%     http://www.eworks.de
%
%%%%%%%%%%%%%%%%%%%%%%%%%%%%%%%%%%%%%%%%%%%%%%%%%%%%%%%%%%%%%%%%%%%%%%%%%%%%%%%%


%%%%%%%%%%%%%%%%%%%%%%%%%%%%%%%%%%%%%%%%%%%%%%%%%%%%%%%%%%%%%%%%%%%%%%%%%%%%%%%%
% Zur Wahl des Seitenverhältnisses bitte einen der beiden folgenden Befehle
% auskommentieren und den ausführen lassen:
% \input{./Ressourcen/Praesentation/Praeambel4zu3.tex} % Seitenverhältnis 4:3
\input{./Ressourcen/Praesentation/Praeambel16zu9.tex} % Seitenverhältnis 16:9
%%%%%%%%%%%%%%%%%%%%%%%%%%%%%%%%%%%%%%%%%%%%%%%%%%%%%%%%%%%%%%%%%%%%%%%%%%%%%%%%


%%%%%%%%%%%%%%%%%%%%%%%%%%%%%%%%%%%%%%%%%%%%%%%%%%%%%%%%%%%%%%%%%%%%%%%%%%%%%%%%
%%%%%%%%%%%%%%%%%%%%%%%%%%%%%%%%%%%%%%%%%%%%%%%%%%%%%%%%%%%%%%%%%%%%%%%%%%%%%%%%
% TUM-Vorlage: Personenspezifische Informationen
%%%%%%%%%%%%%%%%%%%%%%%%%%%%%%%%%%%%%%%%%%%%%%%%%%%%%%%%%%%%%%%%%%%%%%%%%%%%%%%%
%
% Rechteinhaber:
%     Technische Universität München
%     https://www.tum.de
% 
% Gestaltung:
%     ediundsepp Gestaltungsgesellschaft, München
%     http://www.ediundsepp.de
% 
% Technische Umsetzung:
%     eWorks GmbH, Frankfurt am Main
%     http://www.eworks.de
%
%%%%%%%%%%%%%%%%%%%%%%%%%%%%%%%%%%%%%%%%%%%%%%%%%%%%%%%%%%%%%%%%%%%%%%%%%%%%%%%%

% Für die Person anpassen:

\newcommand{\PersonTitel}{Dr.~rer.~nat.}
\newcommand{\PersonVorname}{Leon Adomaitis,}
\newcommand{\PersonNachname}{Alexandros Stathakopoulos}
\newcommand{\PersonStadt}{@Ort@}
\newcommand{\PersonAdresse}{%
    @Adresse@\\%
    @Plz@~\PersonStadt%
}
\newcommand{\PersonTelefon}{@Telefon@}
\newcommand{\PersonFax}{@Fax@}
\newcommand{\PersonEmail}{@E-Mail@}
\newcommand{\PersonWebseite}{@Web@}

\newcommand{\FakultaetAnsprechpartner}{@Ansprechpartner@}
\newcommand{\LehrstuhlName}{@Lehrstuhlname@}

\newcommand{\EinstellungBankName}{Bayerische Landesbank}
\newcommand{\EinstellungBankIBAN}{DE10700500000000024866}
\newcommand{\EinstellungBankBIC}{BYLADEMM}
\newcommand{\EinstellungSteuernummer}{143/241/80037}
\newcommand{\EinstellungUmsatzsteuerIdentifikationsnummer}{DE811193231}

\hyphenation{} % eigene Silbentrennung                    % !!! DATEI ANPASSEN !!!
%%%%%%%%%%%%%%%%%%%%%%%%%%%%%%%%%%%%%%%%%%%%%%%%%%%%%%%%%%%%%%%%%%%%%%%%%%%%%%%%

\renewcommand{\PersonTitel}{}
\newcommand{\Datum}{\today}

\renewcommand{\PraesentationFusszeileZusatz}{}

\title{Checkmate with AI}
\subtitle{Mastering Strategy Through Reinforcement Learning}
\author{Alexandros Stathakopoulos, Mohanad Kandil}
\institute[]{\UniversitaetName}
\date[\Datum]{Heilbronn, 29. Januar 2025}
\subject{}


%%%%%%%%%%%%%%%%%%%%%%%%%%%%%%%%%%%%%%%%%%%%%%%%%%%%%%%%%%%%%%%%%%%%%%%%%%%%%%%%
\input{./Ressourcen/Praesentation/Anfang.tex} % !!! NICHT ENTFERNEN !!!
%%%%%%%%%%%%%%%%%%%%%%%%%%%%%%%%%%%%%%%%%%%%%%%%%%%%%%%%%%%%%%%%%%%%%%%%%%%%%%%%

\graphicspath{ {Ressourcen/_Bilder/} }
%%%%%%%%%%%%%%%%%%%%%%%%%%%%%%%%%%%%%%%%%%%%%%%%%%%%%%%%%%%%%%%%%%%%%%%%%%%%%%%%
% FOLIENSTIL: Standard
\PraesentationMasterStandard

\PraesentationTitelseite % Fügt die Startseite ein

% Slide 1: Agenda
\begin{frame}
    \frametitle{Agenda}
    \begin{enumerate}
        \item Introduction
        \item Model Overview
        \item Algorithm Explanation
        \item Code Structure \& Implementation
        \item Algorithm Performance \& Results
        \item Implications \& Limitations
        \item Conclusion \& Q\&A
    \end{enumerate}
\end{frame}

% Slide 2: Introduction - Background & Motivation
\begin{frame}
    \frametitle{Introduction - Background \& Motivation}
    \begin{itemize}
        \item Problem Statement: What are we solving?
        \item Why is reinforcement learning important?
        \item Brief mention of prior research
        \item Project Goal: Train an AI agent using RL
    \end{itemize}
\end{frame}

% Slide 3: Model Overview
\begin{frame}
    \frametitle{Model Overview}
    \begin{itemize}
        \item What is our RL model trying to achieve?
        \item Type of RL Model: Q-Learning \& Deep Q-Network (DQN)
        \item High-Level Concept: Interaction between Agent and Environment
    \end{itemize}
    \centering
    \includegraphics[scale=0.25]{rl_agent_env}
\end{frame}

% Slide 4: Algorithm Explanation
\begin{frame}
    \frametitle{Algorithm Explanation - Deep Q-Learning}
    \begin{itemize}
        \item Reinforcement learning technique to find optimal action-value function \( Q(s,a) \)
        \item Uses a neural network instead of a Q-table
        \item Action Selection: $\epsilon$-greedy policy for exploration vs exploitation
        \item Experience Replay to improve stability
    \end{itemize}
    \centering
    % \includegraphics[scale=0.2]{dqn_checkers}
\end{frame}

% Slide 5: Code Structure & Implementation
\begin{frame}
    \frametitle{Code Structure \& Implementation}
    \begin{itemize}
        \item Overview of Repository Structure
        \item Key Libraries: Python, TensorFlow/PyTorch, NumPy, OpenAI Gym
        \item Training and Evaluation Pipeline
    \end{itemize}
    \centering
    % \includegraphics[scale=0.2]{code_structure}
\end{frame}

% Slide 6: Algorithm Performance & Results
\begin{frame}
    \frametitle{Algorithm Performance \& Results}
    \begin{itemize}
        \item Performance Metrics: Reward over Episodes
        \item Training Loss over Iterations
        \item Baseline Model Comparison
    \end{itemize}
    \centering
   %  \includegraphics[scale=0.3]{performance_graphs}
\end{frame}

% Slide 7: Implications
\begin{frame}
    \frametitle{Implications}
    \begin{itemize}
        \item Application in Finance, Robotics, Gaming, and Cloud Optimization
        \item Enhancements for real-world decision-making
        \item Future research directions: Combining RL with Transformer models
    \end{itemize}
\end{frame}

% Slide 8: Limitations
\begin{frame}
    \frametitle{Limitations}
    \begin{itemize}
        \item High computational cost
        \item Sample inefficiency
        \item Lack of interpretability
        \item Generalization issues across different environments
    \end{itemize}
\end{frame}

% Slide 9: Conclusion & Summary
\begin{frame}
    \frametitle{Conclusion \& Summary}
    \begin{itemize}
        \item Key Takeaways: Insights from Model Performance
        \item Challenges \& Future Work
    \end{itemize}
    \pause
    \centering
    \textbf{Any Questions?}
\end{frame}



\begin{frame}
	\frametitle{Thank You}
	\vspace{1cm}
	\centering
	\includegraphics{efficiency} \\
	xkcd 1445
\end{frame}

% add here




%%%%%%%%%%%%%%%%%%%%%%%%%%%%%%%%%%%%%%%%%%%%%%%%%%%%%%%%%%%%%%%%%%%%%%%%%%%%%%%%
\end{document} % !!! NICHT ENTFERNEN !!!
%%%%%%%%%%%%%%%%%%%%%%%%%%%%%%%%%%%%%%%%%%%%%%%%%%%%%%%%%%%%%%%%%%%%%%%%%%%%%%%%

