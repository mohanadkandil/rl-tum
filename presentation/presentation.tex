%%%%%%%%%%%%%%%%%%%%%%%%%%%%%%%%%%%%%%%%%%%%%%%%%%%%%%%%%%%%%%%%%%%%%%%%%%%%%%%%
% TUM-Vorlage: Präsentation
%%%%%%%%%%%%%%%%%%%%%%%%%%%%%%%%%%%%%%%%%%%%%%%%%%%%%%%%%%%%%%%%%%%%%%%%%%%%%%%%
%
% Rechteinhaber:
%     Technische Universität München
%     https://www.tum.de
% 
% Gestaltung:
%     ediundsepp Gestaltungsgesellschaft, München
%     http://www.ediundsepp.de
% 
% Technische Umsetzung:
%     eWorks GmbH, Frankfurt am Main
%     http://www.eworks.de
%
%%%%%%%%%%%%%%%%%%%%%%%%%%%%%%%%%%%%%%%%%%%%%%%%%%%%%%%%%%%%%%%%%%%%%%%%%%%%%%%%


%%%%%%%%%%%%%%%%%%%%%%%%%%%%%%%%%%%%%%%%%%%%%%%%%%%%%%%%%%%%%%%%%%%%%%%%%%%%%%%%
% Zur Wahl des Seitenverhältnisses bitte einen der beiden folgenden Befehle
% auskommentieren und den ausführen lassen:
% \input{./Ressourcen/Praesentation/Praeambel4zu3.tex} % Seitenverhältnis 4:3
\input{./Ressourcen/Praesentation/Praeambel16zu9.tex} % Seitenverhältnis 16:9
%%%%%%%%%%%%%%%%%%%%%%%%%%%%%%%%%%%%%%%%%%%%%%%%%%%%%%%%%%%%%%%%%%%%%%%%%%%%%%%%


%%%%%%%%%%%%%%%%%%%%%%%%%%%%%%%%%%%%%%%%%%%%%%%%%%%%%%%%%%%%%%%%%%%%%%%%%%%%%%%%
%%%%%%%%%%%%%%%%%%%%%%%%%%%%%%%%%%%%%%%%%%%%%%%%%%%%%%%%%%%%%%%%%%%%%%%%%%%%%%%%
% TUM-Vorlage: Personenspezifische Informationen
%%%%%%%%%%%%%%%%%%%%%%%%%%%%%%%%%%%%%%%%%%%%%%%%%%%%%%%%%%%%%%%%%%%%%%%%%%%%%%%%
%
% Rechteinhaber:
%     Technische Universität München
%     https://www.tum.de
% 
% Gestaltung:
%     ediundsepp Gestaltungsgesellschaft, München
%     http://www.ediundsepp.de
% 
% Technische Umsetzung:
%     eWorks GmbH, Frankfurt am Main
%     http://www.eworks.de
%
%%%%%%%%%%%%%%%%%%%%%%%%%%%%%%%%%%%%%%%%%%%%%%%%%%%%%%%%%%%%%%%%%%%%%%%%%%%%%%%%

% Für die Person anpassen:

\newcommand{\PersonTitel}{Dr.~rer.~nat.}
\newcommand{\PersonVorname}{Leon Adomaitis,}
\newcommand{\PersonNachname}{Alexandros Stathakopoulos}
\newcommand{\PersonStadt}{@Ort@}
\newcommand{\PersonAdresse}{%
    @Adresse@\\%
    @Plz@~\PersonStadt%
}
\newcommand{\PersonTelefon}{@Telefon@}
\newcommand{\PersonFax}{@Fax@}
\newcommand{\PersonEmail}{@E-Mail@}
\newcommand{\PersonWebseite}{@Web@}

\newcommand{\FakultaetAnsprechpartner}{@Ansprechpartner@}
\newcommand{\LehrstuhlName}{@Lehrstuhlname@}

\newcommand{\EinstellungBankName}{Bayerische Landesbank}
\newcommand{\EinstellungBankIBAN}{DE10700500000000024866}
\newcommand{\EinstellungBankBIC}{BYLADEMM}
\newcommand{\EinstellungSteuernummer}{143/241/80037}
\newcommand{\EinstellungUmsatzsteuerIdentifikationsnummer}{DE811193231}

\hyphenation{} % eigene Silbentrennung                    % !!! DATEI ANPASSEN !!!
%%%%%%%%%%%%%%%%%%%%%%%%%%%%%%%%%%%%%%%%%%%%%%%%%%%%%%%%%%%%%%%%%%%%%%%%%%%%%%%%

\renewcommand{\PersonTitel}{}
\newcommand{\Datum}{\today}

\renewcommand{\PraesentationFusszeileZusatz}{}

\title{Title}
\author{Alexandros Stathakopoulos, Mohanad Kandil}
\institute[]{\UniversitaetName}
\date[\Datum]{Heilbronn, 09. Dezember 2024}
\subject{}


%%%%%%%%%%%%%%%%%%%%%%%%%%%%%%%%%%%%%%%%%%%%%%%%%%%%%%%%%%%%%%%%%%%%%%%%%%%%%%%%
\input{./Ressourcen/Praesentation/Anfang.tex} % !!! NICHT ENTFERNEN !!!
%%%%%%%%%%%%%%%%%%%%%%%%%%%%%%%%%%%%%%%%%%%%%%%%%%%%%%%%%%%%%%%%%%%%%%%%%%%%%%%%


%%%%%%%%%%%%%%%%%%%%%%%%%%%%%%%%%%%%%%%%%%%%%%%%%%%%%%%%%%%%%%%%%%%%%%%%%%%%%%%%
% FOLIENSTIL: Standard
\PraesentationMasterStandard

\PraesentationTitelseite % Fügt die Startseite ein


\begin{frame}
	\frametitle{Introduction}
	\vspace{0.5cm}
	\begin{itemize}
		\item Checkers is a strategy game involving diagonal moves of pieces and captures by jumping over opponent pieces.
		\item The objective of the project is to create an AI player using reinforcement learning.
		\item Techniques: Q-learning and Deep Q-Networks (DQN).
	\end{itemize}
\end{frame}


\begin{frame}
	\frametitle{What is Reinforcement Learning?}
	\vspace{0.5cm}
	\begin{itemize}
		\item A machine learning paradigm focused on training agents to make sequences of decisions.
		\item Key components:
		\begin{itemize}
			\item \textbf{Agent}: Learns to act in an environment.
			\item \textbf{Environment}: The system with which the agent interacts.
			\item  \textbf{Reward}: Feedback signal indicating the success of an action.
		\end{itemize}
		\item Goal: Maximize cumulative rewards over time.
	\end{itemize}
\end{frame}


% Model and algo Slides (WIP)

\begin{frame}
	\frametitle{Deep Q-Learning}
	\vspace{0.5cm}
		\begin{itemize}		
			\item reinforcement learning technique to find the optimal action-value function \texttt{Q(s,a)}
			\item It maps state-action pairs to their expected future rewards
			\item Q-Update rule: $$Q(s,a) = Q(s,a) +  \alpha [r + \gamma  \ max_{a'} \ Q(s',a') - Q(s,a)]$$
		\end{itemize}
		\vspace{1cm}
		\pause
		\centering
		\textbf{Challenge}: Traditional Q-learning fails for large or continuous state spaces
\end{frame}

\begin{frame}
	\frametitle{Deep Q-Learning: Solution }
	\vspace{0.5cm}
	Use a \textbf{Deep Neural Network} (DNN) as a function approximator for $Q(s,a)$ instead of a table.
		\begin{itemize}		
			\item  \textbf{Input}: State $s$, \textbf{Output}: Q-values for all actions $a$
			\item \textbf{Target Network}: stabilizes training by holding fixed weights for a few updates
			\item \textbf{Experience Replay}: samples random batches of past experiences $(s,a,r,s')$ to break correlation in training data
		\end{itemize}
\end{frame}


\begin{frame}
	\frametitle{Deep Q-Learning Workflow }
	\vspace{0.2cm}
	Use a \textbf{Deep Neural Network} (DNN) as a function approximator for $Q(s,a)$ instead of a table.
		\begin{enumerate}		
			\pause
			\item Initialize: \\ Q-Network $Q(s,a;\theta)$ and Target Network $Q'(s,a;\theta^{-})$
			\pause
			\item Action Selection: \\ Use an $\epsilon-$ greedy policy to balance exploration vs. exploitation.
			\pause
			\item Store Experience: \\ Add $(s,a,r,s')$ to a replay buffer
			\pause
			\item Train the Network: \\ 
				\begin{itemize} 
					 \item sample a minibatch from the buffer
					 \item Compute target Q-values: $y = r + \gamma \ max_{a'} \ Q'(s',a';\theta^{-})$
					\item Minimize loss: $L(\theta)=(y-Q(s,a,;\theta))^2$
				\end{itemize}
			\pause
			\item Update Target Network: \\ Periodically copy weights: $\theta^{-} \leftarrow \theta$

		\end{enumerate}
\end{frame}

\begin{frame}
	\frametitle{Key Improvements and Applications}
	\vspace{0.5cm}
	Improvements:
		\begin{itemize}		
			\item \textbf{Double DQN}: Reduces overestimation of Q-values.
			\item\textbf{Dueling DQN}: Separates value and advantage functions.
			\item \textbf{Prioritized Experience Replay}: Samples important experiences more frequently.
		\end{itemize}
	\pause
	Applications:
		\begin{itemize}
			\item Games
			\item robotics
			\item autonomous systems
		\end{itemize}
\end{frame}



% This slide will be before Conc, planning, and project timeline. 

\begin{frame}
	\frametitle{Future Work}
	\vspace{0.5cm}
	\begin{itemize}
		\item Fine-tuning the DQN model for improved decision-making.
		\item Incorporating advanced techniques like Double DQN and Dueling DQN.
		\item Testing against human players to evaluate real-world performance.
		\item Extending the approach to other board games or strategy games (some side-projects).
	\end{itemize}
\end{frame}

% Planning and Timeline

\begin{frame}
	\frametitle{Project Planning & Tiimeline}
	\vspace{0.5cm}
	\begin{itemize}
		\item Iterative Development: 4 sprints $\rightarrow$ 4 weeks
		\item TODO: insert image here
	\end{itemize}
\end{frame}

\PraesentationMasterWeissBlau
\begin{frame}
    \PraesentationUeberschriftZweizeilig{1.}{%
         Introduction }
\end{frame}
\PraesentationMasterStandard

\begin{frame}
    \frametitle{title}
    \vspace{0.5cm}
    lorem ipsum
\end{frame}

% add here




%%%%%%%%%%%%%%%%%%%%%%%%%%%%%%%%%%%%%%%%%%%%%%%%%%%%%%%%%%%%%%%%%%%%%%%%%%%%%%%%
\end{document} % !!! NICHT ENTFERNEN !!!
%%%%%%%%%%%%%%%%%%%%%%%%%%%%%%%%%%%%%%%%%%%%%%%%%%%%%%%%%%%%%%%%%%%%%%%%%%%%%%%%

